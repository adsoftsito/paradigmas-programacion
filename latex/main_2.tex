\documentclass[40pt]{article}
\usepackage{babel}
\usepackage[T1]{fontenc}
\usepackage{textcomp}
\usepackage[utf8]{inputenc} % Puede depender del sistema o editor
\usepackage{enumerate}


\title{\textbf{Universidad Veracruzana} }
\date{\textbf{Facultad de Negocios y Tecnologias} }


\begin{document}
\maketitle
%\section{Integrantes}
\textsf{\Large Experiencia Educativa: Administracion de Software. \\}
 
\maketitle
%\section{Integrantes}
\textsf{\Large Catedratico: Centeno Tellez Adolfo. \\}

\maketitle
%\section{Integrantes}
\textsf{\Large Tema: Plan de Riesgos. \\}

\maketitle
%\section{Integrantes}
\textsf{\Large Integrantes: \\}
\begin{itemize}
    \item Basilio Hernandez Jahaziel.
    \item Hernandez Sanchez Jesus Gabriel.
    \item Jimenez Milan Jose Alfredo.
    \item Perez Castro David.
    \item Zarate Espinosa Jose Pedro.   
\end{itemize}

\maketitle
%\section{Integrantes}
\textsf{\ Grupo: 503 ISW 1° Parcial \\}

\maketitle
%\section{Integrantes}
\textsf{\ Fecha de Entrega: 13 de Octubre del 2020 \\}

\newpage

\maketitle
%\section{Integrantes}
\textsf{\ \\
\textbf{Proposito:}\\
\\
Dentro de nuestro proyecto de software o no solo el nuestro si no de cualquier otro equipo pueden presentarse problemas de riesgo, los cuales son eventos usualmente de impacto negativo, en nuestro proyecto de igual forma existen las condiciones inciertas que a lo largo del desarrollo del proyecto pudieran ocurrir. La elaboración de este documento es para identificar los riesgos que podrían preverse dentro de este proyecto y asi poder prevenir o minimizarlos. \\}
\\
\\
\textbf{Posibles Riesgos:}
\begin{itemize}
    \item Problemas de planificacion del proyecto.
    \item Problemas con el manejo de las tecnologıas.
    \item Problemas con la base de datos.
    \item Problemas del equipo de trabajo.
    \item Problemas con el servidor.   
\end{itemize}



\maketitle
%\section{Integrantes}
\textsf{\ 
\\
\textbf{Responsabilidad:}\\
\\
El líder del proyecto será el principal responsable de todo lo que suceda, el deberá gestionar el plan de trabajo, asi como la calidad del proyecto y que este no sea deficiente, el líder deberá verificar que las actividades se realicen en tiempo y forma, de esta manera deberá tomar medidas estrictas para finalizarlo en el tiempo establecido y que funcione como se estableció. \\}

\maketitle
%\section{Integrantes}
\textsf{\ \\
\\
\textbf{Medición del Riesgo (Probabilidad):} \\}

\begin{tabular}{| c | c |} \hline
\textbf{Medición} & \textbf{Nivel (Alto, Medio, Bajo)} \\ \hline
1 - 3 & Alto \\ \hline
4 - 6 & Medio \\ \hline
7 - 10& Bajo \\ \hline
\end{tabular}

\maketitle
%\section{Integrantes}
\textsf{\ \\
\\
\\
\textbf{Tabla de Riesgos:}\\
\\
Para el apartado de plan de riesgo tenemos contemplado
varios escenarios y circunstancias las cuales podrían afectar
estas son: \\}

\begin{tabular}{| c | c | c |} \hline
\textbf{N.} & \textbf{Tipo de Riesgo} & \textbf{Nivel(Alto, Medio, Bajo)} \\ \hline
1 & Atraso en la fecha de entrega & Medio \\ \hline
2 & Infiltraciones al servidor & Alto \\ \hline
3 & Mala organización & Alto \\ \hline
4 & Fallas en los computadores & Alto \\ \hline
5 & Falta de licencias & Bajo \\ \hline
6 & Disminución de calidad por falta de informacion & Alto \\ \hline
7 & Falta de dominio en los lenguajes & Bajo \\ \hline
8 & Fallos de comunicación grupal & Medio \\ \hline
9 & Problemas con los servicios(luz) & Alto \\ \hline
\end{tabular}

\maketitle
%\section{Integrantes}
\textsf{\ \\
\\
\textbf{Resolucion de Riesgos:}\\
\\
Para este apartado se muestra el numero de riesgo (N.) previsto anteriormente y como se podria resolver en dado caso que llegara a dar problemas, asi como quien seria el responsable de supervisarlo. \\}
\\
1.- Atraso en la fecha de entrega
\begin{itemize}
    \item Se debera crear un calendario, considerando todo detalle que pueda atrasar la entrega.
    \item Supervisar que se cumpla lo establecido por dia.
    \begin{itemize}
      \item Responsable: Lider del Proyecto.
    \end{itemize}  
\end{itemize}
2.- Infltraciones al servidor.
\begin{itemize}
    \item Verificar accesos, asi como fugas de informacion dentro del equipo.
    \item Supervisar ususarios con acceso, minimo una vez a la semana.
    \begin{itemize}
      \item Responsable: Lider del Proyecto.
    \end{itemize}  
\end{itemize}
3.- Mala organización.
\begin{itemize}
    \item Definir un plan de actividades.
    \item Dudas o aclaraciones consultar con el lider.
    \begin{itemize}
      \item Responsable: Lider del Proyecto y Equipo de Trabajo.
    \end{itemize}  
\end{itemize}
4.- Fallas en los computadores.
\begin{itemize}
    \item Definir Recursos necesarios en el plan de trabajo.
    \item Resgistrar anomalias, en reportes diarios.
    \begin{itemize}
      \item Responsable: Lider del Proyecto y Equipo de Trabajo.
    \end{itemize}  
\end{itemize}
5.- Falta de licencias.
\begin{itemize}
    \item Definir programas a utlizar mediante una reunion con el equipo de trabajo.
    \begin{itemize}
      \item Responsable: Lider del Proyecto y Equipo de Trabajo.
    \end{itemize}  
\end{itemize}
6.- Disminución de calidad por falta de informacion.
\begin{itemize}
    \item Contar con la Base de Datos antes de iniciar a trabajar.
    \item Tener alternativas en dado caso de no encontrar toda la informacion exacta.
    \begin{itemize}
      \item Responsable: Lider del Proyecto y Equipo de Trabajo.
    \end{itemize}  
\end{itemize}
7.- Falta de dominio en los lenguajes.
\begin{itemize}
    \item Contemplar pequeños cursos de asesoria.
    \item Compromiso propio por aprender.
    \begin{itemize}
      \item Responsable: Lider del Proyecto y Equipo de Trabajo.
    \end{itemize}  
\end{itemize}
8.- Fallos de comunicación grupal.
\begin{itemize}
    \item Tener canales de comunicacion para que los integrantes puedan interactuar.
    \item Definir medios de comunicacion como: Whatsapp, Zoom, Google Meet, Discord.
    \begin{itemize}
      \item Responsable: Lider del Proyecto y Equipo de Trabajo.
    \end{itemize}  
\end{itemize}
9.- Problemas con los servicios(luz).
\begin{itemize}
    \item Prever en el plan de trabajo, para no afectar entregas.
    \begin{itemize}
      \item Responsable: Lider del Proyecto.
    \end{itemize}  
\end{itemize}




\end{document}